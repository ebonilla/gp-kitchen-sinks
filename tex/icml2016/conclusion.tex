\section{Conclusion}

We have presented multi-task and scalable generalizations of the \egp and \ugp
algorithms introduced by \citet{steinberg-bonilla-nips-2014}, which we refer to
as \eks and \uks respectively. The \uks, like the \ugp is a `black-box' method
in that it requires no gradients of the nonlinearity in the likelihood
function in order to learn parameters/hyperparameters and perform inversions.

We have shown in our experiments that we can achieve similar prediction
performance to the original algorithms, even though we use a \rks-based
Gaussian process approximation to drastically reduce computational complexity
and increase scalability. Furthermore, we demonstrate the \eks and \uks
successfully performing a multi-task, Bayesian seismic inversion -- which is
an ideal use case for these algorithms as a fast and scalable alternative to
MCMC.

As future work we would like to use a stochastic gradient optimiser for
learning all model (hyper)parameters, and also we would like to extend the
posterior representation to a mixture of Gaussians, in a similar fashion as
\citet{nguyen-bonilla-uai-2014} and \citet{gershman-et-al-icml-12}.
