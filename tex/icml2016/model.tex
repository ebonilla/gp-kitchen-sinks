\section{Random Features Approximations}
%
Our starting point to scale up linearized GP models builds upon the work of 
\citeauthor{rahimi-recht-nips-2007} \citeyearpar{rahimi-recht-nips-2007,rahimi-recht-nips-2008},  
who used Bochner's theorem regarding the relationship between 
a kernel (a positive definite function) and the Fourier transform of a non-negative measure. In particular, 
when such a non-negative measure exists, one obtains  Wiener-Khintchine's theorem, which establishes  
 the Fourier duality of the covariance function of a stationary  process and its spectral density:
\begin{align}
	\label{eq:fourier}
	\kernel(\tfourier) &= \int \specfourier(\ffourier) e^{2 \pi i \ffourier^T  \tfourier } d \ffourier \text{,} \\
	\specfourier(\ffourier) &= \int \kernel(\tfourier) e^{- 2 \pi i \ffourier^T \tfourier }  d \tfourier \text{.}
\end{align}
\citeauthor{rahimi-recht-nips-2007}'s  main insight  \citeyearpar{rahimi-recht-nips-2007} 
is that we can approximate the above kernel by explicitly constructing 
``suitable" random features and (Monte Carlo) averaging over samples from $\specfourier(\ffourier)$: 
\begin{equation}
	\label{eq:mcapprox}
	 \kernel(\x-\xprime) = \kernel(\tfourier) 
	\approx \frac{1}{\idim} \sum_{i=1}^{\idim} \singlefeatfunc{\x}{i} \singlefeatfunc{\xprime}{i}  \text{,}
\end{equation}
where $\featfunc{\ins}: \real{d} \to \real{D}$ is a feature map and 
$\singlefeatfunc{\x}{i}$ corresponds to the $i\mth$ component of that map.
%
An example of a feature vector construction in the above approximation is:
\begin{align}
	[\singlefeatfunc{\x}{i} ,\singlefeatfunc{\x}{i+1} ] &= \frac{1}{\sqrt{D}} [ \cos(2 \pi \ffourier_i^T \x) , \sin(2 \pi \ffourier_i^T \ins) ] \text{,}\\
		\ffourier_i & \sim \gausC{\ffourier_i}{\vec{0}, \varfeat \ident{d}} \text{,}
\end{align}
for $i=1, \ldots, D-1$,  which in fact is a mapping into a $2 D-$dimensional feature space. 
\citet{rahimi-recht-nips-2007} used the above feature map to approximate the commonly used 
(isotropic) squared exponential kernel, and showed that such an approximation
converges in expectation 
to the true kernel. They refer to algorithms that use such randomized feature expansions 
as \emph{random kitchen sinks} (\rks).  

If we use \rks bases  such that
$\kfunc{\ins_i}{\ins_j} = 
\expectation[\featfunc{\ins_i}\transpose\featfunc{\ins_j}]$, we can
approximate Gaussian processes that involve nonlinear likelihoods 
with simple linear-in-the-parameters models. 
For our purposes, we are 
interested in using such random feature approximations within a variational inference 
framework in order to estimate the posterior distributions of models of the form given
in Equations \eqref{eq:gp-model-1} and \eqref{eq:gp-model-2}, rather than
approximating a specific covariance function.
 %
%
\section{Multi-task GP Models}
%We have input data $\ins_n \in \real{d}$, which we can transform into features
%$\featfunc{\ins_n} \in \real{D}$ using some function $\featfunc{\cdot}$. We also
%have output targets $\outs_n \in \real{P}$ and our total dataset is 
%$\cbrac{\Outs, \Ins}$ where $\Outs \in \real{N \times P}$ and, $\Ins \in
% \real{N \times d}$.
 Our next step is to approximate our prior over latent functions in Equation \eqref{eq:gp-model-1} 
 using the random feature approximation described above, and to specify 
 our multi-output likelihood in Equation \eqref{eq:gp-model-2}:
 \begin{align}
 \label{eq:prior}
 p(\Weights) &=  \prod^Q_{q=1} \gausC{\weights_q}{\mathbf{0}, \prvar_q \ident{D}} \text{,} \\
  \label{eq:likelihood}
    \probC{\Outs }{\Weights} &=
        \prod^N_{n=1} \gausC{\outs_n}{\nonlin{\Weights\feats_n}, \Lcov}  \text{,}
\end{align}
%
where,  $\feats_n \defeq \featfunc{\ins_n}$  is the 
 $D-$dimensional vector of features corresponding to datapoint $n$; 
$\weights_q \in \real{D}$;
 $\Weights \in \real{Q   \times D}$;
$\prvar_q$ is the prior variance over the weights; 
 and $\Lcov = \diag{\sbrac{\lvar_1, \ldots, \lvar_P}}$ is the  
 noise variance. 
 
Additionally, we note that, as our non-linear
forward model  is $\nonlinf : \real{Q} \to \real{P}$,  
we are effectively  approximating our prior over latent functions as 
 $\latents_q = \Feats\weights_q$, with  $\Feats \defeq \featfunc{\Ins}$
 being the $N \times \idim$  matrix of features evaluated at all the training data. 
 
Having \rks-based approximations allow us to circumvent the inherent
scalability problem in GP models. However, we note that the likelihood model 
in Equation  \eqref{eq:likelihood}, still involves a non linear transformation 
of the corresponding latent functions, which yields, as before, intractable posteriors. 
In order to address this problem, we will build upon the work of 
\citet{steinberg-bonilla-nips-2014}, and develop a variational inference procedure 
that exploits linearization methods around the posterior mean.



 










































