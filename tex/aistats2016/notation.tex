% Acronyms
\newcommand{\rks}{{\sc rks}\xspace}
\newcommand{\gp}{{\sc gp}\xspace}
\newcommand{\egp}{{\sc egp}\xspace}
\newcommand{\ugp}{{\sc ugp}\xspace}
\newcommand{\eks}{{\sc eks}\xspace}
\newcommand{\uks}{{\sc uks}\xspace}
\newcommand{\ut}{{\sc ut}\xspace}



% Simplification of long equations
\newcommand{\gtilden}{\tilde{\nonlinf}_n}
\newcommand{\errorn}{\vecS{\epsilon}_n}


% Parameters of the model
\newcommand{\paramfeat}{\vecS{\theta}_\phi}
\newcommand{\varfeat}{\sigma_{\phi}^2}

% Fourier things
\newcommand{\tfourier}{\vecS{\tau}}   % time space
\newcommand{\ffourier}{\vec{s}} % freq space
\newcommand{\specfourier}{S}


%% Define bracket commands (normal, square and curly).
\newcommand{\brac} [1]  {\ensuremath{\left({#1}\right)}}
\newcommand{\sbrac}[1]  {\ensuremath{\left[{#1}\right]}}
\newcommand{\cbrac}[1]  {\ensuremath{\left\{{#1}\right\}}}
\newcommand{\abrac}[1]  {\ensuremath{\left\langle{#1}\right\rangle}}


%% Symbols

% General
\newcommand{\opt}        {\ensuremath{^{+}}}
\newcommand{\test}       {\ensuremath{^{*}}}
\newcommand{\testT}      {\ensuremath{^{*\top}\!}}
\newcommand{\ttest}      {\ensuremath{^{**}}}
\newcommand{\real}  [1]  {\ensuremath{\mathbb{R}^{#1}}}
\newcommand{\ident} [1]  {\ensuremath{\mathbf{I}_{#1}}}
\newcommand{\all}        {\ensuremath{\star}}
%\newcommand{\all}        {\ensuremath{\raisebox{-0.7ex}{\scalebox{1.8}{$\cdot$}}}}
%\newcommand{\all}        {\ensuremath{{\scalebox{1.8}{$\cdot$}}}}

% Dimensions
\newcommand{\ltdim}     {\ensuremath{Q}} % # of latent functions 
\newcommand{\otdim}     {\ensuremath{P}} % # outputs
\newcommand{\idim}      {\ensuremath{D}} % feature dimension

% Variables
\newcommand{\Latents}   {\ensuremath{\mathbf{F}}}
\newcommand{\latents}   {\ensuremath{\mathbf{f}}}
\newcommand{\latentss}  {\ensuremath{f}}
\newcommand{\latentsq}  {\latents_{\cdot q}}
\newcommand{\latentsn}  {\latents_{n \cdot}}
\newcommand{\hatlatentsn}  {\hat{\latents}_{n \cdot}}
\newcommand{\Outs}      {\ensuremath{\mathbf{Y}}}
\newcommand{\outs}      {\ensuremath{\mathbf{y}}}
\newcommand{\outss}     {\ensuremath{y}}
\newcommand{\Ins}       {\ensuremath{\mathbf{X}}}
\newcommand{\ins}       {\ensuremath{\mathbf{x}}}
\newcommand{\featfunc}[1]  {\ensuremath{\phi\!\brac{{#1}}}}
\newcommand{\singlefeatfunc}[2]  {\ensuremath{\phi_{#2}\!\brac{{#1}}}}
\newcommand{\Feats}     {\ensuremath{\boldsymbol\Phi}}
\newcommand{\feats}     {\ensuremath{\boldsymbol\phi}}
\newcommand{\Weights}   {\ensuremath{\mathbf{W}}}
\newcommand{\weights}   {\ensuremath{\mathbf{w}}}
\newcommand{\weight}    {\ensuremath{{w}}}
\newcommand{\pWeights}  {\ensuremath{\mathbf{M}}}
\newcommand{\pweights}  {\ensuremath{\mathbf{m}}}
\newcommand{\pweight}   {\ensuremath{{m}}}
\newcommand{\Linmat}    {\ensuremath{\mathbf{A}}}
\newcommand{\Lins}      {\ensuremath{a}}
\newcommand{\Linvec}    {\ensuremath{\mathbf{a}}}
\newcommand{\intcpts}   {\ensuremath{b}}
\newcommand{\kernl}     {\ensuremath{k}}
\newcommand{\Kernl}     {\ensuremath{\mathbf{k}}}
\newcommand{\KERNL}     {\ensuremath{\mathbf{K}}}


\newcommand{\prvar}     {\ensuremath{\omega^2}}  % prior variance
\newcommand{\prVar}     {\ensuremath{\boldsymbol\omega}} % prior coovarincae
\newcommand{\Lcov}      {\ensuremath{\mat{\Sigma}}} % Likelihood covariance 
\newcommand{\lvar}      {\ensuremath{\sigma^2}}    % Individual likelihood variances 
\newcommand{\varlatentsn} {\ensuremath{\mathbb{V}[\latentsn] } }  
\newcommand{\varlatentn}[1] {\ensuremath{\mathbb{V}[\latentss_{n#1}] } }  
\newcommand{\sigmafn}[1]{\vec{e}_{n}^{(#1)}}
\newcommand{\mufn}{\vecS{\mu}_n}
\newcommand{\covfn}{\mat{E}_n}
\renewcommand{\matcol}[2]{[#1]_{\cdot,#2}}

\newcommand{\lstd}      {\ensuremath{\sigma}}
\newcommand{\pLatents}  {\ensuremath{\mathbf{M}}}
\newcommand{\platents}  {\ensuremath{\mathbf{m}}}
\newcommand{\pocov}     {\ensuremath{\mathbf{C}}}
\newcommand{\platentss} {\ensuremath{m}}
\newcommand{\pocovs}    {\ensuremath{C}}
\newcommand{\xcov}     {\ensuremath{\boldsymbol\Gamma}}
\newcommand{\khyper}    {\ensuremath{\theta}}
\newcommand{\khypers}   {\ensuremath{\boldsymbol\theta}}
\newcommand{\hyper}     {\ensuremath{\boldsymbol{\theta}}}
\newcommand{\prmean}    {\ensuremath{\boldsymbol{\mu}}}
\newcommand{\Sobs}      {\ensuremath{\mathcal{Y}}}
\newcommand{\Sfunc}     {\ensuremath{\mathcal{M}}}
\newcommand{\scoef}     {\ensuremath{\kappa}}
\newcommand{\Sw}        {\ensuremath{u}}
\newcommand{\Kgain}     {\ensuremath{\mathbf{H}}}
\newcommand{\intcpt}    {\ensuremath{\mathbf{b}}}
\newcommand{\Fengy}     {\ensuremath{\mathcal{F}}}
\newcommand{\LogLike}   {\ensuremath{\mathcal{L}}}
\newcommand{\step}      {\ensuremath{\alpha}}
\newcommand{\Jacob}[1]  {\ensuremath{\mathbf{J}_{#1}}}
\newcommand{\jacob}[1]  {\ensuremath{\mathbf{j}_{#1}}}

% Augmented systems
\newcommand{\augobs}    {\ensuremath{\mathbf{z}}}
\newcommand{\augcov}    {\ensuremath{\mathbf{S}}}
\newcommand{\augLinmat} {\ensuremath{\mathbf{B}}}
\newcommand{\augintcpt} {\ensuremath{\mathbf{c}}}


%% Operations
\newcommand{\transpose}  {\ensuremath{^{\!\top}}}
\newcommand{\inv}        {\ensuremath{^{\text{-}1}}}
\newcommand{\deter}[1]   {\ensuremath{\left|{#1}\right|}}
\renewcommand{\trace}[1]   {\ensuremath{\text{tr}\!\brac{#1}}}
\renewcommand{\diag}[1]    {\ensuremath{\text{diag}\!\brac{#1}}}
\newcommand{\expec}[2]   {\ensuremath{\abrac{#2}_{\!{#1}}}}
\newcommand{\expece}[2]  {\ensuremath{\mathbb{E}_{#1}\!\sbrac{#2}}}
\newcommand{\evar} [2]   {\ensuremath{\mathbb{V}_{#1}\!\sbrac{#2}}}
\newcommand{\KL}[2]      {\ensuremath{\text{KL}\!\sbrac{{#1}\!\parallel\!{#2}}}}
\newcommand{\entropy}[1] {\ensuremath{\mathbb{H}\sbrac{#1}}}
\newcommand{\lnorm}[2]   {\ensuremath{\left\|{#2}\right\|_{{#1}}}}


%% Functions, PDFs etc
\newcommand{\nonlinf}    {\ensuremath{\mathbf{g}}}
\newcommand{\nonlin}[1]  {\ensuremath{\mathbf{g}\!\brac{{#1}}}}
\newcommand{\nonlinfact}[2] {\ensuremath{g_{{#1}}\!\brac{{#2}}}}
\newcommand{\augnonlin}[1] {\ensuremath{h\!\brac{{#1}}}}
\newcommand{\prob}  [1]  {\ensuremath{p\!\brac{#1}}}
\newcommand{\probC} [2]  {\ensuremath{p\!\left({#1}\middle\vert{#2}\right)}}
\newcommand{\qrob}  [1]  {\ensuremath{q\!\brac{#1}}}
\newcommand{\qrobC} [2]  {\ensuremath{q\!\left({#1}\middle\vert{#2}\right)}}
\newcommand{\gaus}  [1]  {\ensuremath{\mathcal{N}\!\brac{#1}}}
\newcommand{\gausC} [2]  {\ensuremath{\mathcal{N}\!\left({#1}\middle\vert{#2}\right)}}
\newcommand{\bern}  [1]  {\ensuremath{\textrm{Bern}\!\brac{#1}}}
\newcommand{\bernC} [2]  {\ensuremath{\textrm{Bern}\!\left({#1}\middle\vert{#2}\right)}}
\newcommand{\kfunc} [2]  {\ensuremath{\kernl\!\brac{{#1}, {#2}}}}
\newcommand{\expon} [2]  {\ensuremath{{#1}\!\times\!10^{#2}}}
\newcommand{\bigo}  [1]  {\ensuremath{\mathcal{O}\!\brac{{#1}}}}


%% Operators
\DeclareMathOperator*{\argmax}{\operatorname*{argmax}}
\DeclareMathOperator*{\argmin}{\operatorname*{argmin}}



\newcommand{\x}{\ins}
\newcommand{\xprime}{\ins^\prime}








