\section{EXPERIMENTS}
\begin{table*}
\caption{Performance of the \eks and \uks methods compared to their GP counterparts (\egp and \ugp) on a range of synthetic benchmarks. 
\gp is the corresponds to the GP analytical solution in the linear case.
}
\begin{center}
\begin{tabular}{c c c c c}
g(f) & Method & SMSE-f* (std) & NLPD-f* (std) &SMSE-g* (std) \\ 
\toprule
linear & \eks & 0.0338 (0.0180) & -1.0382 (0.2664) & 0.0338 (0.0180) \\ 
& \uks & 0.0338 (0.0180) & -1.0383 (0.2663) & 0.0338 (0.0180) \\ 

& \egp & 0.0324 (0.0146) & -1.0105 (0.2995) & 0.0324 (0.0146) \\ 
& \ugp & 0.0345 (0.0176) & -0.9392 (0.4295) & 0.0345 (0.0176) \\ 

poly3 & \eks & 0.0157 (0.0031) & -1.4258 (0.1514) & 0.0045 (0.0008) \\ 
& \uks & 0.0140 (0.0027) & -1.4488 (0.1663) & 0.0045 (0.0009) \\ 

& \egp & 0.0671 (0.0247) & -0.3596 (0.6799) & 0.0164 (0.0053) \\ 
& \ugp & 0.0579 (0.0360) & -0.3544 (0.6156) & 0.0138 (0.0085) \\ 

exp & \eks & 0.0293 (0.0123) & -1.2410 (0.1819) & 0.0132 (0.0045) \\ 
& \uks & 0.0218 (0.0072) & -1.2693 (0.2069) & 0.0116 (0.0022) \\ 

& \egp & 0.0820 (0.0419) & 0.2480 (1.5229) & 0.0367 (0.0223) \\ 
& \ugp & 0.0342 (0.0220) & -1.0185 (0.5768) & 0.0167 (0.0112) \\ 

sin & \eks & 0.0391 (0.0141) & -0.9706 (0.1508) & 0.0342 (0.0090) \\ 
& \uks & 0.0252 (0.0049) & -1.1592 (0.0864) & 0.0261 (0.0055) \\ 

& \egp & 0.0405 (0.0202) & -0.9373 (0.2509) & 0.0374 (0.0228) \\ 
& \ugp & 0.0554 (0.0236) & -0.8044 (0.2175) & 0.0509 (0.0259) \\ 

tanh & \eks & 0.0598 (0.0225) & -1.0856 (0.1461) & 0.0230 (0.0047) \\ 
& \uks & 0.0449 (0.0281) & -1.2922 (0.2163) & 0.0272 (0.0252) \\ 

& \egp & 0.0881 (0.0560) & -0.8520 (0.2398) & 0.0394 (0.0242) \\ 
& \ugp & 0.0504 (0.0290) & -0.8676 (0.2242) & 0.0344 (0.0127) \\ 

\bottomrule
\end{tabular}

\end{center}
\end{table*}
%
\begin{figure*}
\centering
\begin{tabular}{c c c}
\includegraphics[width=0.31\linewidth]{toyData-EKS-SMSE-fstar} &
\includegraphics[width=0.31\linewidth]{toyData-EKS-NLPD-fstar} &
\includegraphics[width=0.31\linewidth]{toyData-EKS-SMSE-gstar} \\
\includegraphics[width=0.31\linewidth]{toyData-UKS-SMSE-fstar} &
\includegraphics[width=0.31\linewidth]{toyData-UKS-NLPD-fstar} &
\includegraphics[width=0.31\linewidth]{toyData-UKS-SMSE-gstar} \\
\end{tabular}
\caption{The performance of the \eks (top) and \uks (bottom) as a function of the number of features used. }
\end{figure*}
